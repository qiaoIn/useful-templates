\chapter{秋招准备}\label{chap:interviews}
\addtocontents{los}{\protect\addvspace{10pt}}

\begin{intro}
整理各面经上的经典面试题,将各个思路进行一个系统化的梳理。
\end{intro}

\section{算法题}

\subsection{判断是否有环?}

\begin{quotation}
{\color{red}如何判断一个链表是否有环?}
\end{quotation}

\begin{enumerate}
\item 最简单也是最直接的想法,从头遍历链表,将已经遍历过的节点信息保存起来,可以
  在遍历一个新节点时,查看该节点之前是否已经遍历过了,增加一个 tag 标记;或者将
  已经遍历过的节点信息使用 HashMap 保存起来,当遍历到一个新节点时,直接查表看之
  前是否已经遍历过该节点。
\item 使用两个指针 slow 和 fast,在刚开始时这两个指针均指向链表首节点,slow 每
  次向前走一步(slow = slow->next;),fast 每次向前走两步
  (fast = fast->next->next;),需要我们自己去判断是否 next 合法,比较 slow
  和 fast 指针指向的节点,如果相同,说明该链表有环;如果不同,继续进行下一次循环
  (slow 向前走一步,fast 向前走两步)。当链表中确实存在环的时候,slow 和 fast
  最终会指向同一个节点;若不存在环,可以通过 fast 达到链表结尾处来结束算法循环。
\end{enumerate}

环形链表的问题可以引申出更多的算法问题,比如:

\begin{enumerate}
  \item [问题一] 在一个有环链表中,如何知道环的长度?如何找出链表的入环点?
  \begin{itemize}
  \item slow 和 fast 第一次相遇可以用于判断链表是否有环,记录下该碰撞点 p,
    slow 和 fast 再次于节点 p 相遇即为环的长度。
  \item 碰撞点 p 到入环点的距离等于头节点到入环点的距离,因此,当 slow 和 fast
    第一次相遇得到碰撞点 p 之后,从碰撞点 p 和头节点开始走(每次走一步),相遇的
    那个点就是入环点。
  \end{itemize}
  \item [问题二] 判断两个单向链表是否相交,如果相交,求出交点。
  \item [问题三] 用于找到链表的中间元素。使用快慢指针(slow 和 fast)可以实现。
\end{enumerate}


\noindent\rule[0.25\baselineskip]{\textwidth}{1pt}

\begin{quotation}
{\color{red}一个有向图用邻接矩阵表示,并且是有权图,现在怎么判断图中有没有环?}
\end{quotation}

拓扑排序的思想可以借鉴。



\subsection{排序算法}\label{subsec:sort-algorithms}


\begin{quotation}
  {\color{red}各种排序算法总结}%
  \footnote{http://www.cnblogs.com/wxisme/p/5243631.html}%
\end{quotation}


\noindent\rule[0.25\baselineskip]{\textwidth}{1pt}

\begin{quotation}
{\color{red}实现快速排序?}
\end{quotation}

\textbf{TODO} 待整理


\noindent\rule[0.25\baselineskip]{\textwidth}{1pt}

\begin{quotation}
  {\color{red}被忽视的 partition 算法}%
  \footnote{https://selfboot.cn/2016/09/01/lost\_partition/}%
\end{quotation}



\subsection{二分查找}\label{subsec:binary-search}

\begin{quotation}
  {\color{red}实现最简单的二分查找,分析二分查找代码实现中需要注意的地方,另外整理可以使用二
  分查找思想来做的一系列题目。}
\end{quotation}


\section{C / C++}\label{sec:c-and-cplusplus}

\subsection{基础知识}

\begin{quotation}
  {\color{red}sizeof 运算符}
\end{quotation}

\verb|sizeof| 运算符返回一条表达式或一个类型名字所占的字节数。 \verb|sizeof| 运
算符满足右结合律,其所得的值是一个 \verb|size_t| 类型的常量表达式(在编译时即可
确定值)。 \verb|sizeof| 运算符的运算对象有两种形式:

\qquad \verb|sizeof (type)| \quad or \quad \verb|sizeof expr|

在第二种形式中, \verb|sizeof| 返回的是表达式结果类型的大小。与众不同的一点是,
\verb|sizeof| 并不实际计算其运算对象的值。

\verb|sizeof| 运算符的结果部分地依赖于其作用的类型:
\begin{itemize}
\item 对 \verb|char| 或者类型为 \verb|char| 的表达式执行 \verb|sizeof| 运算,结
  果得 1;
\item 对引用类型执行 \verb|sizeof| 运算符得到被引用对象所占空间的大小;
\item 对指针执行 \verb|sizeof| 运算得到指针本身所占空间的大小;
\item 对解引用指针执行 \verb|sizeof| 运算得到指针指向的对象所占空间的大小,指针
  不需有效;
\item 对数组执行 \verb|sizeof| 运算得到整个数组所占空间的大小。注意,
  \verb|sizeof| 运算不会把数组转换成指针来处理;
\item 对 \verb|string| 对象或 \verb|vector| 对象执行 \verb|sizeof| 运算只返回该
  类型固定部分的大小,不会计算对象中的元素占用了多少空间。
\end{itemize}


\noindent\rule[0.25\baselineskip]{\textwidth}{1pt}

\begin{quotation}
  {\color{red}C 的结构体和 C++ 结构体的区别}
\end{quotation}

\begin{itemize}
\item C 的结构体内不允许有函数存在,C++ 允许有内部成员函数,且允许该函数是虚函数。
  所以 C 的结构体是没有构造函数、析构函数、和 \verb|this| 指针的。
\item C 的结构体对内部成员变量的访问权限只能是 public,而 C++ 允许
  \verb|public| 、 \verb|protected| 、 \verb|private| 三种。
\item C 语言的结构体是不可以继承的,C++ 的结构体是可以从其他的结构体或者类继承过
  来的。
\end{itemize}

以上都是表面的区别,实际区别就是面向过程和面向对象编程思路的区别:

C 的结构体只是把数据变量给包裹起来了,并不涉及算法。而 C++ 是把数据变量及对这些
数据变量的相关算法给封装起来,并且给对这些数据和类不同的访问权限。C 语言中是没有
类的概念的,但是 C 语言可以通过结构体内创建函数指针实现面向对象思想。


\noindent\rule[0.25\baselineskip]{\textwidth}{1pt}

\begin{quotation}
  {\color{red}C++ struct 和 class 关键字}
\end{quotation}

\begin{itemize}
\item \verb|struct| 和 \verb|class| 都可以用来定义一个类;
\item \verb|struct| 和 \verb|class| 的默认访问权限不太一样;
\item 类可以在它的第一个访问说明符之前定义成员,对这种成员的访问权限依赖于类定义
  的方式。如果使用 \verb|struct| 关键字,则定义再第一个访问说明符之前的成员是
  \verb|public| 的;相反,如果使用 \verb|class| 关键字,则这些成员是
  \verb|private| 的。
\end{itemize}


\noindent\rule[0.25\baselineskip]{\textwidth}{1pt}

\begin{quotation}
  {\color{red}C++ 引用和指针}
\end{quotation}



\noindent\rule[0.25\baselineskip]{\textwidth}{1pt}

\begin{quotation}
  {\color{red}C 语言中的字符串}
\end{quotation}

\noindent 整理自 \textbf{C 程序设计语言(第2版)} 5.5 字符指针与函数,中文版页码 89

字符串常量 是一个字符数组,例如:

\verb|``hello, world\n'';|

\noindent 在字符串的内部表示中,字符数组以空字符 \verb|`\0'| 结尾,程序可以通过检查空字符
找到字符数组的结尾。字符串常量占据的存储单元数也因此比双引号内的字符数大 1。

字符串常量最常用的用法也许是作为函数参数,例如:

\verb|printf(``hello, world.\n'');|

\noindent 当类似于这样的一个字符串出现在程序中时,实际上是通过字符指针访问该字符串的。在上
述语句中, \verb|printf| 接受的是一个指向字符数组第一个字符的指针。也就是说,字符串常量
可通过一个指向其第一个元素的指针来访问。

除了作为函数参数外,字符串常量还有其他用法。假定指针 \verb|pmessage| 的声明如下:

\verb|char *pmessage;|

\noindent 那么,语句

\verb|pmessage = ``hello, world\n'';|

\noindent 将把一个指向该字符数组的指针赋值给 \verb|pmessage| 。该过程并没有进行字符串的复制,而只
是涉及到指针的操作。C 语言没有提供将整个字符串作为一个整体进行处理的运算符。下
面两个定义之间有很大的差别:

\verb|char amessage[] = ``hello, world\n'';  // 定义了一个数组|

\verb|char *pmessage = ``hello, world\n'';  // 定义了一个指针|

上述声明中, \verb|amessage| 是一个仅仅保存足以存放初始化字符串以及空字符
\verb|`\0'| 的一维数组。数组中的单个字符可以进行修改,但 \verb|amessage| 始终指向同一存
储位置。另一方面, \verb|pmessage| 是一个指针,其初值指向一个字符串常量,之后它可以被
修改以指向其他地址,但如果试图通过 \verb|pmessage| 指针修改字符串的内容,结果是没
有定义的(会出现段错误 \verb|Segment Fault| )。

另外,在页码 85 中有描述:

在函数定义中,形式参数 \verb|char s[]| 和 \verb|char *s| 是等价的。我们通常更习
惯于使用后一种形式,因为它比前者更直观地表明了这个参数是一个指针。如果将数组名传
递给函数,函数可以根据情况判定是按照数组处理还是按照指针处理,随后根据相应的方式
操作该参数。为了直观且恰当地描述函数,在函数中甚至可以同时使用数组和指针这两种表
示方法。

另外我们可以从汇编的角度来理解:

第一个字符串是用数组开辟的,它是可以改变的变量。

\begin{sourcecode}[hbtp]
\begin{Verbatim}
movq    L_main.amessage(%rip), %rax
movq    %rax, -22(%rbp)
movl    L_main.amessage+8(%rip), %ecx
movl    %ecx, -14(%rbp)
movw    L_main.amessage+12(%rip), %dx
movw    %dx, -10(%rbp)
movb    \$72, -22(%rbp)  %% TODO
\end{Verbatim}
\caption{amessage 数组表示的汇编代码}
\end{sourcecode}

而第二个字符串则是一个常量,也就是字面值。 \verb|pmessage| 只是指向它的指针而已,
而不能改变指向的内容。

\begin{sourcecode}[hbtp]
\begin{Verbatim}
leaq    L_.str(%rip), %rcx
\end{Verbatim}
\caption{pmessage 指针表示的汇编代码}
\end{sourcecode}

可见用数组和用指针是完全不相同的。

要想通过指针来改变常量是错误,正确的写法应该是用 \verb|const| 指针。

\verb|const char *pmessage = ``hello, world\n'';|

\noindent\rule[0.25\baselineskip]{\textwidth}{1pt}

\begin{quotation}
  {\color{red} C++ 中 string 的使用小结}
\end{quotation}

正是因为 C 风格字符串(以空字符结尾的字符数组)太过复杂难于掌握,不适合大程序的开
发,所以 C++ 标准库定义了一种 \verb|string| 类,定义在头文件 \verb|<string>|。

可以使用 C 风格字符串来初始化 C++ 中的 \verb|string| 对象实例,

\verb|string str(``hello, world\n'');  // correct|

\noindent 但是如果直接将 \verb|string| 类型的对象赋值给 C 风格的字符串的话,编译器会报错,

\verb|char *pstr = str;  // error|

\noindent 但是实际应用中这个问题也难以避免,很多时候还是需要将 \verb|string| 类型的转化为
\verb|char*| 来实现自定义的操作,C++ 标准库也为了和之前用 C 写的程序兼容,于是可
以用 \verb|string| 的 \verb|c_str()| 函数。

\verb|char *pstr = str.c_str();  // error|

\noindent\verb|c_str()| 为了防止意外地修改 \verb|string| 对象,返回的是 \verb|const|
指针,所以上面这段代码是不能被编译的。正确的应该是用 \verb|const| 指针。

\verb|const char *pstr = str.c_str();  // correct|

这个 \verb|c_str()| 方法在 C++ IO 流操作上也被广泛应用。在打开文件时,如果要指定
文件名,可以用 C 风格的字符串。如果用到 \verb|string| 类型的字符串作为文件名时,
就必须调用 \verb|c_str()| 方法将其转换为一个 C 风格字符串。

\begin{sourcecode}[hbtp]
\begin{Verbatim}
string   filename;  //定义文件名称
cin >> filename;
ifstream.open(filename.c_str());  //要使用 C 风格字符串
\end{Verbatim}
\caption{c\_str() 在 C++ IO 流操作上的应用}
\end{sourcecode}

对 \verb|string| 有一定了解后,C++ 标准库还定义的一系列丰富的字符串操作,均基于
\verb|string| 类型。从某一种程度上来说, \verb|string| 就是一种字符容器。

标准库为 \verb|string| 定义了很多方法,包括构造、插入( \verb|insert| )、替换(
\verb|assign| 和 \verb|replace| )、比较( \verb|compare| )、查找( \verb|find| )、
删除( \verb|erase| )、连接( \verb|append| )以及对子串的操作( \verb|substr| ),并
且每一类操作都有很多种重载。%
\footnote{http://www.cnblogs.com/gaojun/archive/2010/09/11/1824016.html}%

\noindent\rule[0.25\baselineskip]{\textwidth}{1pt}

\begin{quotation}
  {\color{red}\verb|std::endl| 与 \verb|`\n'| 有什么区别?在执行的功能上?}
\end{quotation}

\verb|std::endl| 是一个被称为\textbf{操纵符}(manipulator)的特殊值。写入 \verb|std::endl| 的效果是结束
当前行,并将与设备相关联的缓冲区(buffer)中的内容刷到设备中。缓冲刷新操作可以保
证到目前为止程序所产生的所有输出都真正写入输出流,而不是仅停留在内存中等待写入流。

在调试程序时,通常会添加打印语句。这类语句应该保证『一直』刷新流,否则,如果程序
崩溃,输出可能还留在缓冲区中,从而导致关于程序崩溃位置的错误推断。

\verb|`\n'| 为 C++ 语言规定的转义序列之一,\textbf{换行符}

\begin{newnote}[StackExchange 上一个 CodeReview 上的解释]
  The main difference in the two is that std::endl in addition to adding $\backslash$n to
  the stream will flush the stream. The stream will already auto flush at the
  optimal times and forcing a manual flush will only result in a degradation in
  performance.
\end{newnote}

\noindent 何时适用 \verb|std::endl| ,何时适用 \verb|`\n'| 呢?

由于流操作符 \verb|operator<<| 的重载,对于 \verb|`\n'| 和 \verb|``\n''| ,输出效果相同。

对于有输出缓冲的流(例如 \verb|std::cout| 、 \verb|std::clog| ),如果不手动进行缓冲区刷新操作,将在缓冲区
满后自动刷新输出。不过对于 \verb|std::cout| 来说(相对于文件输出流等),缓冲一般体现得并不明
显。但是必要情况下使用 \verb|std::endl| 代替 \verb|`\n'| 一般是个好习惯。

对于无缓冲的流(例如标准错误输出流 \verb|std::cerr| ),刷新是不必要的,可以直接
使用 \verb|`\n'| ,过多的 \verb|std::endl| 是影响程序执行效率低下的因素之一。

由于直接输入/输出和操作系统相关,可能需要切换内核态/用户态,需要一定的时间开销,
频繁地进行操作会极大地降低输入/输出的效率,所以标准库对流的输入/输出操作使用缓冲。
具体来讲,就是在内存中保存一个大小相对固定的区域(缓冲区)用来储存临时的输入或输
出。当必要时,才向系统设备复制缓冲区的内容并清空缓冲区,这个过程称为刷新。

\verb|std::cout << std::endl;|

is equivalent to

\verb|std::cout << `\n' << std::flush;|

So,
\begin{itemize}
\item Use \verb|std::endl| if you want to force an immediate flush to the output.
\item Use \verb|`\n'| if you are worried about performance (which is probably not the
  case if you are using the \verb|<<| operator).
\end{itemize}


\noindent\rule[0.25\baselineskip]{\textwidth}{1pt}

\begin{quotation}
  {\color{red}C++ 中的声明与定义的关系}
\end{quotation}

TODO


\noindent\rule[0.25\baselineskip]{\textwidth}{1pt}

\begin{quotation}
  {\color{red}C++ 中的四个关键字: const / extern / static / volatile }
\end{quotation}

\noindent\textbf{const}

\begin{itemize}
\item 修饰基本数据类型和类类型,说明其为常量,不能修改
\item 修饰指针 \textbf{从右向左进行定义的理解}
  \begin{itemize}
  \item [-] 指向常量的指针(pointer to const) \verb|const| 在 \verb|*| 左边,指
    针指向的是常量,不能使用此指针来改变其所指对象的值
  \item [-] 常量指针(const pointer) \verb|const| 在 \verb|*| 右边,指针本身为
    常量,即不变的是指针本身的值而非指向的那个值。常量指针必须初始化,而且一旦初
    始化完成,则它的值(也就是存放在指针中的那个地址)就不能再改变了
  \end{itemize}
\item 修饰函数参数和返回值,与上一 item 中的『修饰指针』描述相同
\item 修饰类中的数据成员,
\item 修饰类中的成员函数

  \begin{sourcecode}[hbtp]
    \begin{Verbatim}
      struct Sales_data {
        std::string isbn() const { return bookNo; }

        std::string bookNo;
      };
    \end{Verbatim}
    \caption{C++ Primer 上 Sales\_data 类的例子}
  \end{sourcecode}

  类中的成员函数在被调用时,成员函数通过一个名为 \verb|this| 的额外隐式函数来访
  问调用它的那个对象。当我们调用一个成员函数时,用请求该函数的对象地址初始化
  \verb|this| 。例如,如果调用 \verb|total.isbn();| ,则编译器负责把
  \verb|total| 的地址传递给 \verb|isbn| 的隐式形参 \verb|this| ,可以等价地认为
  编译器将该调用重写成了 \verb|Sales_data::isbn(&total);| ,其中,调用
  \verb|Sales_data| 的 \verb|isbn| 成员时传入了 \verb|total| 的地址。

  在成员函数内部,我们可以直接使用调用该函数的对象的成员,而无需通过成员访问运算
  符来做到这一点,因为 \verb|this| 所指的正是这个对象。任何对类成员的直接访问都
  被看作 \verb|this| 的隐式引用,也就是说,当 \verb|isbn| 使用 \verb|bookNo| 时,
  它隐式地使用 \verb|this| 指向的成员,就像写为 \verb|this->bookNo| 一样。

  \verb|this| 形参是隐式定义的,我们可以在成员函数体内部使用 \verb|this|

  \begin{sourcecode}[hbtp]
    \begin{Verbatim}
      struct Sales_data {
        std::string isbn() const { return this->bookNo; }

        std::string bookNo;
      };
    \end{Verbatim}
    \caption{C++ Primer 上 Sales\_data 类的例子,添加上 this}
  \end{sourcecode}

  因为 \verb|this| 总是指向『这个』对象,所以 \verb|this| 是一个常量指针,我们不
  允许改变 \verb|this| 中保存的地址。

  默认情况下, \verb|this| 的类型是指向类类型\textbf{非常量版本}的\textbf{常量指针}。例如在上面的
  \verb|Sales_data| 的成员函数中, \verb|this| 的类型是
  \verb|Sales_data *const| 。尽管 \verb|this| 是隐式的,它仍然需要遵循初始化规则,
  意味着(在默认情况下)我们不能把 \verb|this| 绑定到一个常量对象上。这一情况也
  就使得我们不能在一个常量对象上调用普通的成员函数。

  如果 \verb|isbh| 是一个普通函数而且 \verb|this| 是一个普通的指针参数,则我们应
  该把 \verb|this| 声明成

  \qquad \verb|const Sales_data *const|

  毕竟,在 \verb|isbn|
  的函数体内不会改变 \verb|this| 所指的对象,所以把 \verb|this| 设置为指向常量的
  指针有助于提高函数的灵活性。

  然而, \verb|this| 是隐式的并且不会出现在参数列表中,所以在哪里将 \verb|this|
  声明成指向常量的指针就是我们必须面临的问题。C++ 语言的做法是允许把
  \verb|const| 关键字放在成员函数的参数列表之后,此时,紧跟在参数列表后面的
  \verb|const| 表示 \verb|this| 是一个指向常量的指针。称为\textbf{常量成员函数}
  (const member function)。

  \begin{sourcecode}[hbtp]
    \begin{Verbatim}
      // 伪代码,说明隐式的 this 指针是如何使用的
      // 下面的代码是非法的:因为我们不能显式地定义自己的 this 指针
      // 谨记此处地 this 是一个指向常量的指针,因为 isbn 是一个常量成员
      std::string Sales_data::isbn(const Sales_data *const this) {
        return this->bookNo;
      }
    \end{Verbatim}
    \caption{C++ Primer 上 Sales\_data 类的 isbn 成员函数伪代码}
  \end{sourcecode}

  因为 \verb|this| 是一个指向常量的指针,所以常量成员函数不能改变调用它的对象的
  内容。在上面的例子中, \verb|isbn| 可以读取调用它的对象的数据成员,但是不能写
  入新值。
  
\end{itemize}

\noindent\textbf{extern}

\begin{itemize}
\item 引用其他文件中的 \verb|const| 变量

  当以编译时初始化的方式定义一个 \verb|const| 对象时,编译器将在编译过程中把用到
  该变量的地方都替换成对应的值。为了执行此替换,编译器必须知道变量的初始值。如果
  程序包含多个文件,则每个用了 \verb|const| 对象的文件都必须得能访问到它的初始值
  才行。要做到这一点,就必须在每一个用到变量的文件中都有对它的定义。为了支持这一
  用法,同时避免对同一变量的重复定义,默认情况下, \verb|const| 对象被设定为仅在
  文件内有效。当多个文件中出现了同名的 \verb|const| 变量时,其实等同于在不同文件
  中分别定义了独立的变量。

  某些时候有这样一种 \verb|const| 变量,它的初始值不是一个常量表达式,但又确实
  有必要在文件间共享。解决办法是,对于 \verb|const| 变量不管是声明还是定义都添加
  \verb|extern| 关键字,这样只需要定义一次就可以了:

  \begin{sourcecode}[hbtp]
    \begin{Verbatim}
      // example.cc 定义并初始化了一个常量,该常量能被其他文件访问
      extern const int bufSize = fcn();
      // example.h 头文件
      extern const int bufSize;  // 与 example.cc 中定义的 bufSize 是同一个
    \end{Verbatim}
    \caption{在多个文件之间共享 const 对象,必须在变量的定义之前添加 extern}
  \end{sourcecode}

\item C++ 程序有时候需要调用其他语言编写的函数,最常见的是调用 C 语言编写的函数。
  需使用 \verb|extern| 关键字
\end{itemize}

\noindent\textbf{static}

C++ 的 \verb|static| 有两种用法:面向过程程序设计中的 \verb|static| 和面向对象程
序设计中的 \verb|static| 。前者应用于普通变量和函数,不涉及类;后者主要说明
\verb|static| 在类中的作用。

\noindent\textbf{面向过程设计中的 static}
\begin{itemize}
\item [1] \textbf{静态全局变量}

  在全局变量前,加上关键字 \verb|static| ,该变量就被定义成为一个静态全局变量。
  静态全局变量有以下特点:
  \begin{itemize}
  \item 该变量在全局数据区分配内存;
  \item 未经初始化的静态全局变量会被程序自动初始化为 0(自动变量的值是随机的,除
    非它被显式初始化);
  \item 静态全局变量在声明它的整个文件都是可见的,而在文件之外是不可见的
  \end{itemize}

  静态变量都在全局数据区分配内存,包括后面将要提到的静态局部变量。对于一个完整的
  程序,在内存中的分布情况如下图:

  \qquad 代码区

  \qquad 全局数据区

  \qquad 堆区

  \qquad 栈区

  一般程序的由 \verb|new| 产生的动态数据存放在堆区,函数内部的自动变量存放在栈区。
  自动变量一般会随着函数的退出而释放空间,静态数据(即使是函数内部的静态局部变量)
  也存放在全局数据区。全局数据区的数据并不会因为函数的退出而释放空间。定义全局变
  量就可以实现变量在文件中的共享,但定义静态全局变量还有几个好处:1)静态全局变
  量不能被其它文件所用;2)其它文件中可以定义相同名字的变量,不会发生冲突。
  
\item [2] \textbf{静态局部变量}

  在局部变量前,加上关键字 \verb|static| ,该变量就被定义成为一个静态局部变量。
  通常,在函数体内定义了一个变量,每当程序运行到该语句时都会给该局部变量分配栈内
  存。但随着程序退出函数体,系统就会收回栈内存,局部变量也相应失效。但有时候我们
  需要在两次调用之间对变量的值进行保存。通常的想法是定义一个全局变量来实现。但这
  样一来,变量已经不再属于函数本身了,不再仅受函数的控制,给程序的维护带来不便。
  静态局部变量正好可以解决这个问题。静态局部变量保存在全局数据区,而不是保存在栈
  中,每次的值保持到下一次调用,直到下次赋新值。

  静态局部变量有以下特点:
  \begin{itemize}
  \item 该变量在全局数据区分配内存;
  \item 静态局部变量在程序执行到该对象的声明处时被首次初始化,即以后的函数调用不
    再进行初始化;
  \item 静态局部变量一般在声明处初始化,如果没有显式初始化,会被程序自动初始化为
    0;
  \item 它始终驻留在全局数据区,直到程序运行结束。但其作用域为局部作用域,当定义
    它的函数或语句块结束时,其作用域随之结束
  \end{itemize}
  
\item [3] \textbf{静态函数}

  在函数的返回类型前加上static关键字,函数即被定义为静态函数。静态函数与普通函数
  不同,它只能在声明它的文件当中可见,不能被其它文件使用。定义静态函数的好处:1)
  静态函数不能被其它文件所用;2)其它文件中可以定义相同名字的函数,不会发生冲突

\end{itemize}

\noindent\textbf{面向对象的 static 关键字(类中的 static 关键字)}
\begin{itemize}
\item [4] \textbf{静态数据成员}

  在类内数据成员的声明前加上关键字 \verb|static| ,该数据成员就是类内的静态数据
  成员。静态数据成员有以下特点:
  \begin{itemize}
  \item 对于非静态数据成员,每个类对象都有自己的拷贝。而静态数据成员被当作是类的
    成员。无论这个类的对象被定义了多少个,静态数据成员在程序中也只有一份拷贝,由
    该类型的所有对象共享访问。也就是说,静态数据成员是该类的所有对象所共有的。对
    该类的多个对象来说,静态数据成员只分配一次内存,供所有对象共用。所以,静态数
    据成员的值对每个对象都是一样的,它的值可以更新;
  \item 静态数据成员存储在全局数据区。静态数据成员定义时要分配空间,所以不能在类
    声明中定义。
  \item 静态数据成员和普通数据成员一样遵从 \verb|public| ,\verb|protected| ,
    \verb|private| 访问规则;
  \item 因为静态数据成员在全局数据区分配内存,属于本类的所有对象共享,所以,它不
    属于特定的类对象,在没有产生类对象时其作用域就可见,即在没有产生类的实例时,
    我们就可以操作它;
  \item 静态数据成员初始化与一般数据成员初始化不同。静态数据成员初始化的格式为:

    \qquad <数据类型><类名>::<静态数据成员名> = <值>
    
  \item 类的静态数据成员有两种访问形式:

    \qquad <类对象名>.<静态数据成员名> or <类类型名>::<静态数据成员名>

    如果静态数据成员的访问权限允许的话(即 \verb|public| 的成员),可在程序中,
    按上述格式来引用静态数据成员;
    
  \item 静态数据成员主要用在各个对象都有相同的某项属性的时候。比如对于一个存款
    类,每个实例的利息都是相同的。所以,应该把利息设为存款类的静态数据成员。这
    有两个好处,1)不管定义多少个存款类对象,利息数据成员都共享分配在全局数
    据区的内存,所以节省存储空间;2)一旦利息需要改变时,只要改变一次,则所
    有存款类对象的利息全改变过来了;
  \item 同全局变量相比,使用静态数据成员有两个优势:1)静态数据成员没有进入程
    序的全局名字空间,因此不存在与程序中其它全局名字冲突的可能性;2)可以实现
    信息隐藏。静态数据成员可以是 \verb|private| 成员,而全局变量不能
  \end{itemize}
  
\item [5] \textbf{静态成员函数}

  与静态数据成员一样,我们也可以创建一个静态成员函数,它是为类类型提供服务而不是
  为某一个类的具体对象服务。静态成员函数与静态数据成员一样,都是类的内部实现,属
  于类定义的一部分。普通的成员函数一般都隐含了一个 \verb|this| 指针,
  \verb|this| 指针指向类的对象本身,因为普通成员函数总是具体的属于某个类的具体对
  象的。通常情况下, \verb|this| 是缺省的。但是与普通函数相比,静态成员函数由于
  不是与任何的对象相联系,因此它不具有 \verb|this| 指针。从这个意义上讲,它无法
  访问属于类对象的非静态数据成员,也无法访问非静态成员函数,它只能访问其余的静态
  成员(包括静态成员变量和静态成员函数)。

  关于静态成员函数,可以总结为以下几点:
  \begin{itemize}
  \item 出现在类体外的函数定义不能指定关键字 \verb|static| ;
  \item 静态成员之间可以相互访问,包括静态成员函数访问静态数据成员和访问静态成员
    函数;
  \item 非静态成员函数可以任意地访问静态成员函数和静态数据成员;
  \item 静态成员函数不能访问非静态成员函数和非静态数据成员;
  \item 由于没有 \verb|this| 指针的额外开销,因此静态成员函数与类的全局函数相比
    速度上会有少许的增长;
  \item 调用静态成员函数,可以用成员访问操作符( \verb|.| )和( \verb|->| )为一个
    类的对象或指向类对象的指针调用静态成员函数,也可以直接使用如下格式:

    \qquad <类名>::<静态成员函数名>(<参数表>)

    调用类的静态成员函数
  \end{itemize}  
\end{itemize}

\verb|static| 关键字总结:
\begin{itemize}
\item 修饰局部变量,把局部变量从栈区移动到静态数据区,改变声明周期,不改变作用域
\item 修饰全局变量,改变作用域,由全工程可见变成本源文件可见
\item 修饰函数,与修饰全局变量相同
\item 修饰类变量,是类的一部分,只有唯一一份副本
\item 修饰类成员函数,没有 \verb|this| 指针,只能访问静态成员(包括静态成员变量
  和静态成员函数)
\end{itemize}

另外可以参看 C++ 中 \verb|static| 作用和使用方法%
\footnote{https://blog.csdn.net/artechtor/article/details/2312766}%


\noindent\textbf{volatile}

\begin{itemize}
\item 谈谈 C/C++ 中的 \verb|volatile| %
  \footnote{https://liam0205.me/2018/01/18/volatile-in-C-and-Cpp/}%
  \quad 在多线程环境下,不应该假设 \verb|volatile| 能够解决多线程中的某些问题,
  而应该选用 C++11 中的互斥量和条件变量,分别在头文件 \verb|mutex| 和
  \verb|condition_varialbe| 。

  \begin{sourcecode}[hbtp]
    \begin{Verbatim}
      // global shared data
      std::mutex m;                   // #include <mutex>
      std::condition_variable cv;     // #include <condition_variable>
      bool flag = false;
      
      thread1() {
        flag = false;
        Type* value = new Type(/* parameters */);
        thread2(value);
        std::unique_lock<std::mutex> lk(m);
        cv.wait(lk, [](){ return flag; });
        apply(value);
        lk.unlock();
        thread2.join();
        if (nullptr != value) { delete value; }
        return;
      }
      
      thread2(Type* value) {
        std::lock_guard<std::mutex> lk(m);
        // do some evaluations
        value->update(/* parameters */);
        flag = true;
        cv.notify_one();
        return;
      }
    \end{Verbatim}
    \caption{使用互斥锁和条件变量实现多线程之间的数据同步,伪代码}
  \end{sourcecode}
  
\item C/C++ 中 \verb|volatile| 关键字详解 %
  \footnote{https://www.cnblogs.com/yc\_sunniwell/archive/2010/07/14/1777432.html}%
\item 易变性:两条语句,如果存在变量重用的话下一条语句会重新从内存中读取;
\item 不可优化:保证含有 \verb|volatile| 的指令一定会被执行,而不会被激进的消除;
\item 顺序性:保证 \verb|volatile| 变量间的顺序性,编译器不会乱序优化,但 CPU 还
  是可能乱序发射执行。
\end{itemize}


\noindent\rule[0.25\baselineskip]{\textwidth}{1pt}

\begin{quotation}
  {\color{red}C++ 中的四种强制类型转换: static\_cast / dynamic\_cast / const\_cast / reinterpret\_cast }
\end{quotation}

一个命名的强制类型转换具有如下形式:

\verb|cast-name<type>(expression);|

\noindent 其中, \verb|type| 是转换的目标类型, \verb|expression| 是要转换的值。
如果 \verb|type| 是引用类型,则结果是左值。 \verb|cast-name| 是 \verb|static_cast| 、
\verb|dynamic_cast| 、 \verb|const_cast| 和 \verb|reinterpret_cast| 中的一种。
\verb|cast-name| 指定了执行的是那种转换。

\noindent\textbf{static\_cast}

\begin{itemize}
\item 允许执行任意的隐式转换和相反的转换动作(无论是否被允许)。也可以用于基础类
  型之间的标准转换;
\item 如类的指针,允许子类类型的指针转换成父类类型的指针(这是一个有效的隐式转
  换),同时,也能够执行相反动作,转换父类为它的子类。
\end{itemize}

\noindent\textbf{dynamic\_cast}

只用于对象的指针和引用。当用与多态类型时,它允许任意的隐式类型转换以及相反过程。
不过,与 \verb|static_cast| 不同,在后一种情况里(注:即隐式转换的相反过程),
\verb|dynamic_cast| 会检查操作是否有效。也就是说,它会检查转换是否会返回一个被
请求的有效的完整对象。 \verb|dynamic_cast| 支持运行时类型识别。 

\noindent\textbf{const\_cast}

操纵传递对象的 \verb|const| 属性,或者是设置或者是移除;一般用于强制消除对象的常
量性。一旦我们去掉了某个对象的 \verb|const| 性质,编译器就不再阻止我们对该对象进
行写操作了。如果对象本身不是一个常量,使用强制类型转换获得写权限是合法的行为。然
而如果对象是一个常量,再使用 \verb|const_cast| 执行写操作就会产生未定义的后果。

只有 \verb|const_cast| 能改变表达式的常量属性,使用其他形式的命名强制类型转换改
变表达式的常量属性都将引发编译器错误。

\noindent\textbf{reinterpret\_cast}

转换一个指针为其他类型的指针,或者转换一个指针为整数类型,操作结果是简单的从一个
指针到别的指针的值得二进制拷贝。

\noindent\textbf{注意:}强制类型转换干扰了正常的类型检查,因此我们强烈建议程序员
避免使用强制类型转换。

\subsection{C++ 语言特性}

\begin{quotation}
  {\color{red}STL}%
  \footnote{STL 源码剖析,侯捷 \quad 著}%
\end{quotation}

\noindent\textbf{各类容器的底层实现}

\begin{center}
  \begin{tabular}{lp{24em}}
    \hline
    容器 & 实现 \\
    \hline
    \textbf{\ttfamily vector} & 数组,支持快速随机访问 \\
    \textbf{\ttfamily deque}  & deque 是一个双端队列(double-ended queue),也是在堆中保存内容的。它的保存形式如下: [heap1] -> [heap2] -> [heap3] 每个堆保存好几个元素,然后堆与堆之间有指针指向,看起来像是 list 和 vector 的结合 \\
    \textbf{\ttfamily list}   & 双向链表,支持快速增删 \\
    \textbf{\ttfamily stack} & 用 list 或 deque 实现,封闭头部 \\
    \textbf{\ttfamily queue} & 用 list 或 deque 实现,封闭头部 \\
    \textbf{\ttfamily priority\_queue} & 底层数据结构一般为 vector 为底层容器,堆 heap 为处理规则来管理底层容器实现 \\
    \textbf{\ttfamily set} & 红黑树,有序,不重复 \\
    \textbf{\ttfamily multiset} & 红黑树,有序,可重复 \\
    \textbf{\ttfamily map} & 红黑树,有序,不重复 \\
    \textbf{\ttfamily multimap} & 红黑树,有序,可重复 \\
    \textbf{\ttfamily unordered\_set} & 拉链 hash 表,无序,不重复 \\
    \textbf{\ttfamily unordered\_multiset} & 拉链 hash 表,无序,可重复 \\
    \textbf{\ttfamily unordered\_map} & 拉链 hash 表,无序,不重复 \\
    \textbf{\ttfamily unordered\_multimap} & 拉链 hash 表,无序,可重复 \\ 
    \hline
  \end{tabular}
\end{center}

\begin{center}
  \textbf{注:}拉链 \verb|hash| 表,当桶内的链表过长影响性能时,将存储的结构由链
  表换为红黑树,来提高效率。
\end{center}

\begin{itemize}
\item [1] 顺序容器
  \begin{itemize}
  \item \textbf{\ttfamily vector}
    \begin{enumerate}
    \item 在内存中有连续的存储空间,支持快速随机访问,插入删除效率慢,空间分配以
      2 的倍数动态增长,每次内存空间不足需要先分配新的 2 倍大小的空间,然后把原
      先空间中的元素拷贝到新空间中;
    \item \verb|clear| 不会消除 \verb|vector| 在内存中的元素,可以改用
      \verb|swap| (临时变量)。
    \end{enumerate}
  \item \textbf{\ttfamily deque} \quad 支持两端插入数据,内存空间分布是小片的连续,小片之
    间用链表相连,重新分配空间不需要拷贝原有元素。
  \item \textbf{\ttfamily list} \quad 双向链表,内存空间不连续,通过指针进行数据访问,随机
    存储低效,插入删除低效。
  \end{itemize}
\item [2] 关联容器
  \begin{itemize}
  \item \textbf{\ttfamily set / multiset(允许重复)} \quad 内部有序,元素唯一,无法直接存储
    元素,只能通过迭代器间接存取。
  \item \textbf{\ttfamily map / multimap(允许重复)} \quad 内部有序,红黑树中的每个节点在不
    保存数据时,占用 16 个字(父指针、左右孩子指针、标识红黑色的枚举值),占内存
    较大。
  \end{itemize}
\item [3] 容器适配器
  \begin{itemize}
  \item \textbf{\ttfamily queue} \quad \verb|deque| 基础上封装,先进先出。
  \item \textbf{\ttfamily stack} \quad \verb|deque| 基础上封装,先进后出。
  \end{itemize}
\end{itemize}

\noindent\textbf{算法}

如 \verb|sort| 、 \verb|search| 等,模板函数, \verb|[first, last)| 的迭代器
区间。

\noindent\textbf{迭代器}

一种『范型指针』,所有容器都有自己的迭代器,只有容器本身知道如何遍历自己的元素,
对指针的 \verb|+| 、 \verb|*| 、 \verb|-| 、 \verb|->| 等操作进行重载。

迭代器失效的总结:
\begin{center}
  \begin{tabular}{lp{24em}}
    \hline
    容器 & 失效情况 \\
    \hline
    \textbf{\ttfamily vector} & vector 的迭代器在内存重新分配时失效(它所指向的元素在该操作的前后不再相同)。当把超过 capacity() - size() 个元素插入 vector 中时,内存会重新分配,所有的迭代器都将失效;否则,指向当前元素以后的任何迭代器都将失效。当删除元素时,指向被删除元素之后的任何元素的迭代器都将失效。 \\
    \textbf{\ttfamily deque}  & 增加任何元素都将使 deque 的迭代器失效。在 deque 的中间删除元素将使迭代器失效。在 deque 的头或尾删除元素时,只有指向该元素的迭代器失效。 \\
    \textbf{\ttfamily list}   & 增加任何元素都不会使迭代器失效。删除元素时,除了指向当前被删除元素的迭代器外,其他迭代器都不会失效。 \\
    \textbf{\ttfamily set} & 如果迭代器所指向的元素被删除,则该迭代器失效。其他任何增加、删除元素的操作都不会使迭代器失效。 \\
    \textbf{\ttfamily map} & 如果迭代器所指向的元素被删除,则该迭代器失效。其他任何增加、删除元素的操作都不会使迭代器失效。 \\
    \hline
  \end{tabular}
\end{center}

\noindent\textbf{仿函数}

仿函数是一种重载了 \verb|operator()| 的类,使类的行为类似函数。

\noindent\textbf{适配器}

用来修饰容器接口、迭代器接口或仿函数接口;如 \verb|stack| 、 \verb|queue| 用来修
饰 \verb|deque| 。

\noindent\textbf{空间适配器}

为容器进行空间配置和管理。


\noindent\rule[0.25\baselineskip]{\textwidth}{1pt}

\begin{quotation}
  {\color{red}虚函数}
\end{quotation}

虚函数是一种在基类定义为 \verb|virtual| 的函数,并在一个或多个派生类中再定义的函
数。

\noindent\textbf{实现}

\begin{itemize}
\item 通过 \verb|vtbl| 和 \verb|vptr| 来实现。每个声明或继承了虚函数的类,都有自
  己的 \verb|vtbl| ,该类的每个对象都有指向 \verb|vtbl| 的 \verb|vptr|
\item 继承:如果子类覆盖了父类的虚函数,将被放到虚函数表中原来父类虚函数的位置;在多
  继承的情况下,对于派生类,几重继承,就会有几个虚函数表,这些表按照派生的顺序依
  次排列,如果子类改写了父类的虚函数,那么就会用子类自己的虚函数覆盖虚函数表的相
  应的蚊子,如果子类有新的虚函数,那么就添加到第一个虚函数表的末尾。
\end{itemize}

需要添加一个图片。

\noindent\textbf{性能}

通过 \verb|vptr| 类型找到 \verb|vtbl| ,在 \verb|vtbl| 中找到指针指向的函数。单
继承性能差不多,多继承会慢(多个 \verb|vbtl| 中查找)。

\noindent\textbf{占用空间}

虚函数表会增加类的体积,在类的继承过程中,每个子类都有父类的虚函数表。对于子类对
象,但继承情况下会多一个 \verb|vptr| 指针的体积(4 字节);多继承情况下会多 N 个
\verb|vptr| 指针的体积(4 $\times$ N 字节)。

\begin{sourcecode}[hbtp]
  \begin{Verbatim}
    class A {
     public:
      virtual void funa();
      virtual void funb();
      void func();
      static void fund();
      static int si;
     private:
      int i;
      char c;
    };

    Q: sizeof(A);
    A: 12(32 位系统)或 16(64 位系统)
    // 一个指针(4 或者 8) + int(4)+ char(1) 计算对齐
  \end{Verbatim}
  \caption{虚函数表占用空间示例}
\end{sourcecode}

关于类占用的内存空间,有以下几点需要注意:
\begin{enumerate}
\item 如果类中含有虚函数,则编译器需要为类构建虚函数表,类中需要存储一个指针指向
  这个虚函数表的首地址,注意不管有几个虚函数,都只建立一张表,所有的虚函数地址都
  存在这张表里,类中只需要一个指针指向虚函数表首地址即可;
\item 类中的静态成员是被类所有实例所共享的,它不计入 \verb|sizeof| 计算的空间;
\item 类中的普通函数或静态普通函数都存储在栈中,不计入 \verb|sizeof| 计算的空间;
\item 类成员采用字节对齐的方式分配空间
\end{enumerate}


\noindent\textbf{纯虚函数}

在基类中没有定义,但要求任何派生类都要定义自己的实现方法。

\noindent\textbf{虚析构函数}

用一个基类的指针删除一个派生类的对象时,派生类的析构函数会被调用。只有当一个类被
用来作为基类的时候,才把析构函数写成虚函数。

\noindent\textbf{虚函数类型}

\begin{itemize}
\item \textbf{不能成为虚函数的函数类型}
  \begin{itemize}
  \item 内联函数,编译时展看,必须有实体;
  \item 静态成员函数,属于类,必须有实体;
  \item 构造函数, \verb|vtbl| 在类被创建后产生,如果析构函数是虚函数,那么构造
    函数自己无法创建类的实例,也就没有 \verb|vtbl| 去访问。
  \end{itemize}
\item \textbf{可以成为虚函数的函数类型}
  \begin{itemize}
  \item 析构函数
  \end{itemize}
\end{itemize}


\noindent\rule[0.25\baselineskip]{\textwidth}{1pt}

\begin{quotation}
  {\color{red}多态是怎么实现的?}
\end{quotation}

C++ 的多态性分为静态多态和动态多态:
\begin{itemize}
\item 静态多态性:编译期间确定具体执行哪一项操作,主要是通过函数重载和运算符重载
  来实现的;
\item 动态多态性:运行时确定具体执行哪一项操作,主要是通过虚函数来实现的。
\end{itemize}




\noindent\rule[0.25\baselineskip]{\textwidth}{1pt}

\begin{quotation}
  {\color{red}右值}%
  \footnote{https://liam0205.me/2016/12/11/rvalue-reference-in-Cpp/}%
\end{quotation}



\noindent\rule[0.25\baselineskip]{\textwidth}{1pt}

\begin{quotation}
  {\color{red}智能指针}
\end{quotation}

\begin{itemize}
\item 谈谈代理类%
  \footnote{https://liam0205.me/2017/11/26/surrogate-in-Cpp/}%
\item 谈谈智能指针:原理及其实现%
  \footnote{https://liam0205.me/2018/01/13/smart-pointer/}%
\item 谈谈 C++ 中的右值引用%
  \footnote{https://liam0205.me/2016/12/11/rvalue-reference-in-Cpp/}%
\end{itemize}



\subsection{一些实现}\label{subsec:some-implementation}

\begin{itemize}
\item 无锁数据结构%
  \footnote{http://blog.jobbole.com/90811/}%
\item C++ 并发编程实战
\item ZeroMQ 实现中用到的无锁队列
\item ps-lite 中实现的 \verb|SArray| ,类似于标准库中的 \verb|vector|
\end{itemize}


\noindent\rule[0.25\baselineskip]{\textwidth}{1pt}

\begin{quotation}
  {\color{red}C++ 实现单例模式}
\end{quotation}


\noindent\rule[0.25\baselineskip]{\textwidth}{1pt}

\begin{quotation}
  {\color{red}C++ 使用两个栈实现一个队列}%
  \footnote{http://www.cnblogs.com/wanghui9072229/archive/2011/11/22/2259391.html}%
\end{quotation}


\section{操作系统}\label{sec:os-related}

\subsection{进程、线程与协程}


\subsection{select / poll / epoll}


\subsection{RPC}

\begin{itemize}
\item 深入理解 RPC%
  \footnote{https://juejin.im/entry/57c866230a2b58006b204712}%
\item 谁能用通俗的语言解释一下什么是 RPC 框架?%
  \footnote{https://www.zhihu.com/question/25536695}%
\end{itemize}


\section{面试回顾及整理}\label{sec:interviews-and-reviews}

\subsection{微信支付}

\noindent\textbf{一面} 2018 年 8 月 16 日 17:30

\begin{enumerate}
\item 个人自我介绍?
\item 如何查看一个进程的父进程号?

  ps -ef 就可以看到,平常多使用的是 ps aux。这两者有何区别呢?
  
\item 在一个目录下有很多个其他文件(包含文件和目录),其中有一个文件的文件名是乱
  码,应该执行什么样的 Linux 命令将这个文件名乱码的文件给删除掉?

  Unix/Linux 系统内部不使用文件名,而使用 inode 号来识别文件。对于系统来说,文件
  名只是 inode 号便于识别的别称或者绰号。表面上,用户通过文件名,打开文件。实际
  上,系统内部这个过程分成三步:首先,系统找到这个文件名对应的 inode 号;其次,
  通过 inode 号,获取 inode 信息;最后,根据 inode 信息,找到文件数据所在的
  block,读出数据。

  使用 ls -i 命令,可以看到文件名对应的 inode 号。有时,文件名包含特殊字符,无法
  正常删除。这时,直接删除 inode 节点,就能起到删除文件的作用。

  这里两步来走:

  \begin{enumerate}
  \item [1)] 查看包含乱码的文件名的 inode 号

    ls -il
    
  \item [2)] 利用 inode 号,执行 find 和 xargs 命令删除该文件

    \begin{itemize}
    \item [-] find ./ -inum 5725772 -exec rm -i {} \;
    \item [-] rm -i `find ./ -inum 5725772'

      这两个方法都比较好理解,通过 ls -i 这个命令获得要删除文件的 inode 号,
      然后使用 find 命令的 -inum 选项查找对应 inode 号的文件名,然后将文件名
      通过 find 的 -exec 参数或者通 `' 反引号传递给 rm 命令

    \item [-] find ./ -inum 5725772 | xargs rm -i

      在使用 find 命令的 -exec 选项处理匹配到的文件时,find 命令将所有匹配到
      的文件一起传递给 exec 执行。但有些系统对能够传递给 exec 的命令长度有限制,
      这样在 find 命令运行几分钟之后,就会出现溢出错误。错误信息通常是“参数列太
      长”或“参数列溢出”。这就是 xargs 命令的用处所在,特别是与 find 命令一起
      使用。
    \end{itemize}
  \end{enumerate}

  
\item 当在一个浏览器的搜索框中输入网址时会发生怎样的一系列动作?涉及到哪些技术?

  \begin{itemize}
  \item [-] IP $\rightarrow$ TCP $\rightarrow$ HTTP
  \item [-] DNS 域名解析服务(域名 $\rightarrow$ IP)、ARP 地址解析协议
    (IP $\rightarrow$ MAC Address)、ANT(子网 $\leftrightarrow$ 内网)
  \item [-] 负载均衡、一致性哈希,CDN
  \item [-] etc$\cdots$
  \end{itemize}
  
\item 在实现 C++ 类时,如果我们不自己实现拷贝构造函数和复制构造函数,编译会为我
  们自动生成,那么,在什么情况下需要自己手动实现这两个函数呢? %% TODO Chapter13
\item STL 使用时,在什么情况下选择使用 std::vector,在什么情况下选用
  std::list?%% TODO Chapter3.3
\item std::vector 对象是如何增长的? %% TODO Chapter9
\item 如何查找 5 亿个元素的 Top-100(最大的 100 个元素)?

  有两种思路:

  \begin{itemize}
  \item [1)] 求最大的 Top-K 使用最小堆,求最小的 Top-K 使用最大堆

    \begin{itemize}
    \item [\textbf{思路}] 使用堆排序来实现 Top-K 的思路很直接。最大(小)堆就
      是其堆顶元素最大(小),比所有的子节点都要大(小)。求最大的 Top-K 个元素,
      构建最小堆(堆中共有 K 个元素),堆顶的元素就是堆中最小的元素。堆就保存的
      是当前扫描到的元素中的 Top-K 个了,怎么理解呢?现在我们从文件头开始读取文
      件内容,读入文件中的前 K 个元素构建好最小堆(当前最小堆中包含元素为 $A_1$
      -- $A_K$,为当前读取得所有 K 个元素的 Top-K 组成的堆,假设文件中包含的元
      素以 A 作为标记,其索引是元素被读取到的顺序),接着读取文件中剩余的元素,例
      如接下来读取到的是文件中的 (K+1)个元素 $A_{K+1}$,将 $A_{K+1}$ 与已构建最
      小堆的堆顶元素进行比较,如果 $A_{K+1}$较小,那就不用入堆,因为当前读到的所
      有元素 $A_1$ -- $A_{K+1}$的 Top-K (最大)就是最小堆中的 K 个元素了;如
      果$A_{K+1}$ 较大,将堆顶元素移出,将 $A_{K+1}$ 入堆(这里需要删除堆顶元素
      并插入一个新的元素,需要调整堆,仍然为最小堆)。以上,即,最小堆中的 K 个元
      素始终保存着当前已经扫描过的元素的 Top-K(最大)元素,新扫描到的元素比堆顶
      小(比当前最大的 K 个元素中的最小的还要小)就不需要入堆了,新扫描到的元素
      比堆顶大,将堆顶元素移出,新扫描到的元素入堆,调整仍未最小堆。
    \item[\textbf{速记口诀}] 最小的 K 个用最大堆,最大的 K 个用最小堆。
    \item[\textbf{时间复杂度}] $\mathcal{O}(nlogK)$
    \item[\textbf{适用场景}] 实现的过程中,我们先用前 K 个数建立了一个堆,然后
      扫描所有元素来维护这个堆。这种做法带来了三个好处:
      
      \begin{itemize}
      \item [1)] 不会改变数据的输入顺序(按顺序读的);
      \item [2)] 不会占用太多的内存空间(事实上,一次只读入一个数,内存只要求能
        容纳前 K 个数即可);
      \item [3)] 由于 2),决定了它特别适合处理海量数据。
      \end{itemize}

      这三点,也决定了它最优的适用场景。

    \end{itemize}
    
  \item [2)] 借鉴快速排序的 partition 函数的思路,进行折半

    \begin{itemize}
    \item [\textbf{思路}] 
    \item [\textbf{时间复杂度}]  $\mathcal{O}(n)$
    \item [\textbf{适用场景}] 对照着堆排的解法来看,partition 函数会不断地交换
      元素的位置,所以它肯定会改变数据输入的顺序;既然要交换元素的位置,那么所有元
      素必须要读到内存空间中,所以它会占用比较大的空间,至少能容纳整个数组;数据
      越多,占用的空间必然越大,海量数据处理起来相对吃力。但是,它的时间复杂度很
      低,意味着数据量不大时,效率极高。
    \end{itemize}
    
  \end{itemize}
  
\item 你有什么问题想问我的么?

  微信支付是如何日活跃量如此巨大的访问的,有用到哪些技术?NJX、RPC 模块、多线程
  库、手动实现了协程库(rountine)等等
  
\end{enumerate}

\noindent\textbf{准备一下}

针对你简历上写的、你曾经做过的、你最值得说道的一个项目或者是一个需求或者是一次优
化不断追问。作为应聘者你一定要充分做好这方面的准备。建议首先要在你的简历中有意突
出这个点,让面试官注意到然后想要来问你。接下来你要针对这个点去准备材料,可以从以
下几个方面去准备:

\begin{itemize}
\item 为什么要做这个?
\item 你是如何做这个的?
\item 期间你遇到了什么问题?你又是如何解决的?
\item 你做完这个带来了什么实际的效果?
\item 你做完之后呢?有继续优化和改进吗?
\item 等等
\end{itemize}

\noindent\textbf{WXG 面委-技术 GM 面试} 2018 年 9 月 5 日 17:30



\endinput





