\chapter{算法与数据结构}\label{chap:algorithms}
\addtocontents{los}{\protect\addvspace{10pt}}

\begin{intro}

\end{intro}

\section{基本数据结构}\label{sec:primary-data-structures}

\begin{itemize}
\item 算法第四版 C++ 实现
\item 浅谈算法和数据结构%
  \footnote{https://www.cnblogs.com/yangecnu/category/548373.html}%
\end{itemize}

\section{分而治之}\label{sec:divide-and-conquer}


\section{贪心}\label{sec:greedy}

\subsection{}

\subsection{}


\section{动态规划}\label{sec:dynamic-programming}

\begin{quotation}
以下内容整理自:

\begin{itemize}
  \item 知乎上问题『什么是动态规划?动态规划的意义是什么?』
  \item 算法导论 Chapter 15 % TODO
  \item 算法 Chapter 6 % TODO 
\end{itemize}
\end{quotation}

\subsection{动态规划的本质}

动态规划的本质,是对问题\textbf{状态的定义}和\textbf{状态转移方程的定义}。引用维基百科:
\begin{quotation}
\textbf{Dynamic programming} is a method for solving a complex problem by \textbf{breaking it down
into a collection of simpler subproblems.}
\end{quotation}
动态规划就是通过\textbf{拆分问题},定义问题状态和状态之间的关系,使得问题能够以递推(或者说分治)的方式去解决。
\textbf{如何拆分问题},才是动态规划的核心。而\textbf{拆分问题},靠的就是\textbf{状态的定义}和\textbf{状态转移方程的定义}。

\subsection{状态的定义}

首先来看一个动态规划的教学必备题:
\begin{quotation}
给定一个数列,长度为$N$,求这个数列的最长上升(递增)子序列(LIS)的长度。以 1\quad7\quad2\quad8\quad3\quad4 为例,这个数列的
最长递增子序列是 1\quad2\quad3\quad4,长度为 4;次长的为 3,包括 1\quad7\quad8、1\quad2\quad3、1\quad3\quad4。
\end{quotation}
要解决这个问题,首先要\textbf{定义这个问题}和\textbf{这个问题的子问题}。
有人可能会问了,题目都已经在这了,还需定义这个问题吗?需要,原因就是这个问题在字面上看,找不出子问题,而没有子问题,这个题目就
没办法解决。

所以,重新定义这个问题:
\begin{quotation}
给定一个数列,长度为$N$,设 $F_{k}$ 为:以数列中第 $k$ 项结尾的最长递增子序列的长度。求 $F_{1}, \cdots, F_{N}$ 中的最大值。
\end{quotation}
显然,这个新问题与原问题等价。而对于 $F_{k}$ 来讲,$F_{1}, \cdots, F_{k-1}$ 都是 $F_{k}$ 的子问题:因为以第 $k$ 项结尾的
最长递增子序列(以下称 LIS),包含着以第 $1, \cdots, k-1$ 中某项结尾的 LIS。

上述的新问题 $F_{k}$ 也可以叫做状态,定义中的『$F_{k}$ 为数列中第 $k$ 项结尾的 LIS 的长度』,就叫做对状态的定义。之所以把 $F_{k}$
叫做『状态』而不是『问题』,一是因为避免跟原原题中的『问题』混淆,二是因为这个新问题是数学化定义的。

对状态的定义只有一种吗?当然不是。我们甚至可以二维的,以完全不同的视角定义这个问题:
\begin{quotation}
给定一个数列,长度为$N$,设 $F_{i, k}$ 为:在前 $i$ 项中的长度为 $k$ 的最长递增子序列中,最后一位的最小值。其中 $1 \leq k \leq N$ 。
若在前 $i$ 项中,不存在长度为 $k$ 的最长递增子序列,则 $F_{i, k}$ 为正无穷。求最大的 $x$,使得 $F_{N, x}$ 不为正无穷。
\end{quotation}
上述的 $F_{i, k}$ 就是状态,定义中的『$F_{i, k}$ 为:在前 $i$ 项中的长度为 $k$ 的最长递增子序列中,最后一位的最小值』就是对
状态的定义。

\subsection{状态转移方程}

上述状态定义好之后,状态和状态之间的关系式,就叫做\textbf{状态转移方程}。

比如,对于 LIS 问题,第一种定义:
\begin{quotation}
设 $F_{k}$ 为:以数列中第 $k$ 项结尾的最长递增子序列的长度。
\end{quotation}
设 $A$ 为题中数列,状态转移方程为:
\begin{quotation}
$F_{1} = 1$(根据状态定义导出边界情况)

$F_{k} = max(F_{i} + 1 | A_{k} > A_{i}, i \epsilon (1\cdots k-1)) (k > 1)$
\end{quotation}
用文字解释一下是:以第 $k$ 项结尾的 LIS 的长度是,保证第 $i$ 项比第 $k$ 项小的情况下,以第 $i$ 项结尾的 LIS 长度加一的最大值,取遍 $i$
的所有值($i < k$)。

第二种定义:
\begin{quotation}
设 $F_{i, k}$ 为:在前 $i$ 项中的长度为 $k$ 的最长递增子序列中,最后一位的最小值。
\end{quotation}
设 $A$ 为题中数列,状态转移方程为:
\begin{quotation}
若 $A_{i} > F_{i-1, k-1}$,$F_{i, k} = min(A_{i}, F_{i-1, k})$;否则 $F_{i, k} = F_{i-1, k}$。
\end{quotation}
这里可以看出,状态转移方程,就是定义了问题和子问题之间的关系。可以看出,状态转移方程就是带有条件的递推式。

\subsection{动态规划迷思}

\noindent\textbf{缓存,重叠子问题,记忆化}

这三个名词,都是在阐述递推式求解的技巧。以 Fibonacci 数列为例,计算第 100 项的时候,需要计算第 99 项和第 98 项;在计算第 101 项的时候,需要第
100 项和第 99 项。这时候你还需要重新计算第 99 项吗?不需要,你只需要在第一次计算的时候把它记下来就可以了。上述的需要再次计算的『第 99 项』,就叫
『重叠子问题』。如果没有计算过,就按照递推式计算,如果计算过,直接使用,就像『缓存』一样,这种方法,叫做『记忆化』,这是递推式求解的技巧。这种技巧,
通俗的说叫『花费空间来节省时间』。\textbf{都不是动态规划的本质,不是动态规划的核心。}

\noindent\textbf{递归}

递归是递归是求解的方法,连技巧都算不上。

\noindent\textbf{无后效性,最优子结构}

上述的状态转移方程中,等式右边不会用到下标大于左边 $i$ 或者 $k$ 的值,这是『无后效性』的通俗上的数学定义,符合这种定义的状态定义,我们可以说她他具有
『最有子结构』的性质,在动态规划中我们要做的,就是找到这种『最优子结构』。

在对状态和状态转移方程的定义过程中,满足『最优子结构』是一个隐含条件(否则根本定义不出来)。对状态和和『最优子结构』的进一步理解\textbf{TODO}

\begin{newnote}[对状态和和『最优子结构』的进一步理解]
动态规划是对于\textbf{某一类问题}的解决方法!重点在于如何鉴定『某一类问题』是动态规划可解的,而不是纠结解决方法是递归还是递推!

怎么鉴定动态规划可解的一类问题,需要从计算机是怎么工作的说起。计算机的本质是一个状态机,内存里存储的所有数据构成了当前的状态,CPU 只能利用当前的状态
计算出下一个状态(不要纠结硬盘之类的外部存储,就算考虑它们也只是扩大了状态的存储容量而已,并不能改变下一个状态只能从当前状态计算出来这一条铁律)。
当你企图使用计算机解决一个问题是,其实就是在思考如何将这个问题表达成状态(用哪些变量存储哪些数据)以及如何在状态中转移(怎样根据一些变量计算出
另一些变量)。所以,所谓的空间复杂度就是为了支持你的计算所必需存储的状态最多有多少,所谓时间复杂度就是从初始状态到达最终状态中间需要多少步!
\end{newnote}

一个问题是该用递推、贪心、搜索还是动态规划,完全是由这个问题本身阶段间状态的转移方式决定的!
\begin{itemize}
	\item 每个阶段只有一个状态 $\rightarrow$ 递推;
	\item 每个阶段的最优状态都是由上一个阶段的最优状态得到的 $\rightarrow$ 贪心;
	\item 每个阶段的最优状态是由之前所有阶段的状态的组合得到的 $\rightarrow$ 搜索;
	\item 每个阶段的最优状态可以从之前某个阶段的某个或某些状态直接得到而不管之前这个状态是如何得到的 $\rightarrow$ 动态规划。
\end{itemize}

『每个阶段的最优状态可以从之前某个阶段的某个或某些状态直接得到』,这个性质叫做\textbf{最优子结构};而不管之前这个状态是如何得到的,这个性质
叫做\textbf{无后效性}。

需要注意的是,一个问题可能有多种不同的状态定义和状态转移方程定义,存在一个有后效性的定义,不代表该问题不适合动态规划。动态规划方法要寻找符合『最优子结构』
的状态和状态转移方程的定义,在找到之后,这个问题就亦可以以『记忆化地求解递推式』的方法来解决。而寻找到的定义,才是动态规划的本质。即,\textbf{动态规划
是寻找一个对问题的观察角度,让问题能够以递推(或者说分治)的方式去解决,寻找看问
题的角度,才是动态规划中最耀眼的宝石!}


\section{回溯法}\label{sec:back-tracking}



\section{分支限界法}\label{sec:branch-and-bound}




\endinput
