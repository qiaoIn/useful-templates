\chapter{杂的言}\label{chap:misc}
\addtocontents{los}{\protect\addvspace{10pt}}

\begin{intro}
这里主要进行一些杂项的整理,对于自己接触到的一些感兴趣的公众号上的文章,或者平常看书的一些感悟,还有就是实际工作学习中遇到的一些问题的整理,即杂的言。

可以每天早上半小时、晚上半小时整理公众号上看到的知识点,生成 notes.pdf,然后发送到手机上进行 review 修改,自己的一些知识点也可以整理在那里。
一天整理一点,以每天的日期

每周对这七天的整理进行总结

\end{intro}

\section{每月知识点归档}\label{sec:everymonth-notes}

每个月\textbf{月末最后一周}将知识点做一个归档,例如加到之前已有的章节中,或者是新建一个章节。

\section{每周整理为一些集中的点}\label{sec:everyweek-notes}

每一个 review 之后将 \ref{sec:everyday-notes} 节的内容进行一个整理,梳理到不同的 subsection 中, subsection 以各项技术为标题,
例如缓存 \ref{subsec:cache}。

\subsection{缓存}\label{subsec:cache}


\section{每日杂项整理}\label{sec:everyday-notes}

将每天自己学习到的不能归类的杂项整理在这里,以时间为 subsection 的标题。每周来一个统一的 review,在固定的时间,\textbf{每周六},另外每天晚上坐车的时候
也可以 review 一下。

\subsection{20180710}\label{subsec:date-20180710}

\begin{itemize}
    \item Git 抓取某一非 master 分支代码?

    git branch -r 查看远程分支

    或者 git branch -a 查看所有分支

    其后直接 git checkout [分支名] 就可以了

    \item CLion 中比较两个 git 分支的差异:选中工程 -> 右键 -> git -> compare with branch
\end{itemize}

\subsection{20180711}\label{subsec:date-20180711}

\begin{itemize}
	\item 自定义 Git - 配置 Git https://git-scm.com/book/zh/v1
    \item git fetch 的含义
    \item Change the email address for a git commit. https://gist.github.com/trey/9588090
    \item 
\end{itemize}



\endinput
